\chapter{Background and Motivation}
\label{chap2}

\section{Environmental Perspective}

In recent decades, there has been significant concern about various forms of pollution on our planet. Air pollution, in particular, has been a major issue for many years, especially in the most industrially advanced countries. A significant consequence of air pollution is the increase in particulate matter in the atmosphere, also known as PM\textsubscript{2.5}. These particles can lead to various health problems, including cardiovascular, respiratory and neurodegenerative diseases \cite{RecentInsightsParticulateMatter}. This, in conjunction with the increasing global temperatures \cite{effectofheatandairpollution}, has shown that even a slight increase in particulate matter pollution can have significant effects. Numerous advanced countries and regions are making strides towards reducing air pollution by not only mitigating 
 \cite{EuropeEnvironment} the factors causing this pollution, but also by enhancing the techniques for monitoring and analyzing air quality.

A prime example of efforts to address this situation is South Korea \cite{IQAir} which is using advanced monitoring and notification technologies to alert its citizens when high levels of air pollution are detected. This includes the involvement of technology companies, like IQAir \cite{IQAir}, which provides high-performance real-time air quality monitoring systems to help combat the country's serious air pollution issues. When air pollution levels surge, citizens are alerted to limit outdoor activities and are advised to use masks capable of filtering particulate matter as a protective measure. Moreover, the initiative to adopt more advanced technologies for tracking the sources and impacts of air pollution has already started to show results, as evidenced by the reduction in the annual PM\textsubscript{2.5} concentration level \cite{AirPollutionControl}, in the past years starting from 2019 to 2022. 

\section{IoT Perspective}

Since such efforts require substantial government and enterprise funds, there has been a push for more accessible, simplified and cost-effective solutions. Internet of Things (IoT), coupled with the widespread impact of the internet on our daily lives \cite{madakam2015internet,ImpactofIoT-BasedSmartCities} over the past decade, has emerged as a central instrument for these accessible solutions. The IoT principles has evolved over the years, finding its place in numerous implementations and applications \cite{ImpactofIoT-BasedSmartCities,7562698}. It utilizes a broad spectrum of device-to-device and device-to-Cloud implementations \cite{IotArchitecture}. Device-to-Device (D2D) communication, devices directly exchange data, optimizing local processing efficiency, but limitations arise in terms of range, scalability challenges and potential security vulnerabilities. Device-to-Cloud (D2C) implementations leverage cloud resources for scalability and cost-efficiency, offering centralized management, but may introduce latency, security concerns and dependence on reliable internet connectivity \cite{CloudIoT}. However, challenges in Device-to-Cloud interactions require careful selection and configuration of the frameworks for a robust implementation. This has paved the way for innovative business models and applications, promoting both technological advancement and economic growth in cities and regions that have embraced this approach in recent years \cite{rose2015internet,7300835}.

A bright example in our local environment in the city of Volos, Greece is GreenyourAir \cite{GreenyourAir}. This IoT-based application employs simple and cost-effective sensors to measure the PM\textsubscript{2.5} concentration level, along with the temperature and humidity, initially in the city of Volos and more recently in other areas as well. It is, of course, utilizing Cloud infrastructure to store and process the real time data coming from affordable sensors. The use of smaller, cost-effective sensors, instead of large stations, facilitates the expansion of the existing network in a cost-efficient way. This approach broadens the availability and accessibility of air quality information to the public. In our thesis, we utilize a portion of their measurements as our data ingestion source, providing real-world data for our application.


\clearpage
\section{Open-source Approach}\label{open_source_approach}
To proceed further, it's essential to understand the significance of open source in general , as the entire expansion of our SYNAISTHISI platform is based on the implementation of open-source tools. Open source projects possess huge influence in the development process of IoT, web applications and projects in general, as they raise a collaborative environment where developers contribute their expertise, leading to robust and innovative solutions \cite{BenefitsofOpen-SourceSoftwareforDevelopers}. This continuous integration accelerates the development process, as issues are identified and resolved swiftly. Furthermore, open source projects provide a lot of learning opportunities for developers through a wide range of levels, while also ensuring transparency, as the source code is openly available for review, improvement and modification. Additionally, transparency enhances trust and allows for the customization of applications to meet specific needs, many large companies \cite{OpenSourceContributorIndex} from various industries to contribute to open source projects, aiming to develop frameworks and services in a more efficient and cost-effective manner. In the IoT field, open source projects have been key in pushing progress, providing many libraries, frameworks and tools \cite{IntelOpensource} that simplify the development process, which combined with the ongoing support for these tools, enhances the scalability and future-readiness. In conclusion, most of the technologies we utilize for our implementation are originated from well-know open-source projects, that provide advanced development tools and frameworks, enabling the appreciation to the significance and value of open source for both small and large scale applications.

\section{SYNAISTHISI Platform}

The primary platform that we are enhancing in this thesis is SYNAISTHISI \cite{Synaisthisi}. It offers the interoperability, scalability and interconnectivity that our application requires. There is a wide variety of supported protocols in the platform, but in this instance, we are primarily focusing on interconnecting Kafka and its associated tools with MQTT and AMQP protocols, through the integrated RabbitMQ brokers that are already supported by the platform. This strategy aligns with the principles outlined in the open-source approach section (cf. Sec. \hyperref[open_source_approach]{2.3}), as we aim to expand our application and integrate it with services that continue to receive support from their development communities, making our project more open and future-proof. In addition, the containerized nature of the services within the SYNAISTHISI platform facilitates an easy way to upgrade the existing infrastructure with Kafka tools and enables a smooth transition to a production environment. Thus, we preserve the IoT perspective of our application and expand it with advanced yet open-source services.

\section{Kafka Integration Large Scale Event Streaming }

Real-time event streaming and processing, involves transmitting events and messages in a distributed system, allowing for reliable, low-latency and high-throughput data transfer \cite{EventStreamingPlatform}. 
Apache Kafka \cite{WhatIsApacheKafka}, is a widely used open-source stream-processing software, designed for large-scale data collection, processing, storage and analysis \cite{KafkaApachePoweredBy}. Since its launch in 2011 \cite{WhatIsApacheKafka}, the top three tech companies - Amazon, Microsoft and Google - have developed their own large-scale event streaming services: Amazon Kinesis \cite{AmazonKinesis}, Azure Event Hubs \cite{AzureEventHubs} and Google Pub/Sub \cite{GooglePub/Sub}, respectively, inspired by the Apache Kafka project. We note that these are paid services on fully managed Cloud provider platforms \cite{CloudNativeApplications}. However, by utilizing Kafka, we gain access to many features typically found in such paid services, along with the large-scale, high-volume and fault-tolerant data ingestion solutions that Kafka offers. Therefore, with the proper configuration and understanding of our application's needs, we can still leverage these capabilities using this widely adopted and contributor-maintained open-source service.