% Αγγλική Περίληψη

\begin{abstracteng}\setlanguage{american}
In this thesis, we design and develop an event-driven Ιnternet of Things (IoT) Air Quality monitoring application, that tracks PM\textsubscript{2.5} particles concentration levels, humidity and temperature. We utilize advanced cloud computing tools, and expand the SYNAISTHISI platform.

Environmental air pollution has been a big concern for a lot of people lately. Our goal is to provide a user-friendly solution for anyone willing to monitor air quality in a given region. Any citizen should be informed of their local air quality, and beyond that, should be able to create their own tools to monitor it in an easy and accessible way. Starting with this basic principle, we utilize new industry standard cloud tools and services, in combination with existing Internet of Things protocol standards, creating a universally available and highly scalable application. 

In recent years, there have been a lot of attempts to create simple affordable Internet of Things solutions for a plethora of problems -in our case air pollution- by using simple and accessible implementations. In our approach, we attempt to go a step further and implement a more complex in its development, but quite straight forward on its operation and capabilities, cloud based solution. In line with the principles of most IoT applications, we have designed our solution to be both accessible and user-friendly.

Additionally, we extend an existing research-oriented IoT platform by integrating the Apache Kafka protocol. Apache Kafka provides high throughput, low latency, and high availability capabilities to our application. The structure of our stack revolves around evolving the capabilities of our SYNAISTHISI platform, by utilizing new cloud tools that can provide backward compatibility with other protocols -MQTT, RabbitMQ-, that are currently widely used by IoT edge devices and merge them with the advanced capabilities of Apache Kafka and its counterparts.

Our focus is on environmental measurements and distributing this data to any system that the SYNAISTHISI platform operators wish to utilize. The addition of Apache Kafka and its accompanying tools expands the potential for a wide variety of cloud computing applications. Thus, by developing our air quality application, we are able to explore the technologies that are utilized in our platform. These technologies can be implemented in various services and IoT applications, aligning with the primary vision of the SYNAISTHISI platform.

\vspace*{\fill}
\noindent{\large\bf{Keywords:}}\\ 
Ιnternet of Things, PM\textsubscript{2.5} particles, cloud tools, SYNAISTHISI platform,air pollution, air quality, Apache Kafka
\end{abstracteng}